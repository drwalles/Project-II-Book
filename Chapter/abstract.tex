\pagebreak

\section*{\underline{Abstract}}

This document is an example of one-page abstract for the book of abstracts
of the Modelica Conference 2011. The full conference proceedings will be only published
electronically on a memory stick and on the Web. However, for the conference
attendant's convenience, a smaller book of abstracts will be printed and made available
in the conference kit given to all participants.

The goal of the book of abstracts is to provide the audience with some more information
about the papers being presented, so they can choose which presentations to attend during parallel
sessions. To this end, you can of course re-use the abstract provided when submitting the paper on 
the conference management system. However, consider that this page will be a sort of
advertisement for your presentation, which will be read during coffe-breaks to decide
if your presentation is worth listening to. Therefore, you can try to make it more 
attractive and informative. For this purpose, you can also 
include figures, such as Fig. \ref{fig5}, and a few optional references, such as \cite{dna}
or \cite{Modelica}.
\begin{figure}[h]
%uncomment next line to include a graphic file
%\centerline{\includegraphics[width=6cm, angle=-90]{fig5.eps}}
%and comment out next line
\centerline{\framebox[6cm]{\rule{0cm}{3.5cm} figure example}}
\caption{Structure of the DNA double helix}
\label{fig5}
\end{figure}

The one-page abstract should be written in English and prepared as a one-page, A4 PDF file (size 210$\times$297 mm) using the Times font, 18 pt for the title, 13 pt for the authors' names and affiliations, 
and 11 pt for the text. The left and right margins should be 3.5 cm, the top margin 3 cm and the bottom margin
2 cm. Please do not exceed these limits, or your PDF file will not be accepted. The page will then be
scaled down slightly for printing. We advise you to use the {\LaTeX} and MS Word templates provided on the conference website in order to ensure the correct formatting.

Please also note that this abstract must be uploaded on the conference management
system no later than January 17, 2011, and that at least one author should have registered by that time,
for the final paper to be included in the conference programme and proceedings.





